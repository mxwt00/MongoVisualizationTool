\begin{abstract}

    \iffalse
    Ziel der Kurzfassung ist es, einen (eiligen) Leser zu informieren, so 
    dass dieser entscheiden kann, ob der Bericht für ihn hilfreich ist oder 
    nicht (neudeutsch: Management Summary). Die Kurzfassung gibt daher eine 
    kurze Darstellung

    \begin{itemize}
    \item des in der Arbeit angegangenen Problems
    \item der verwendeten Methode(n)
    \item des in der Arbeit erzielten Fortschritts.
    \end{itemize}

    Dabei sollte nicht auf die Struktur der Arbeit eingegangen werden, also 
    Kapitel~\ref{cha:grundlagen} etc. denn die Kurzfassung soll ja gerade 
    das Wichtigste der Arbeit vermitteln, ohne dass diese gelesen werden muss.
    Eine Kapitelbezogene Darstellung sollte sich in Kapitel~%
    \ref{cha:einleitung} unter Vorgehen befinden.

    Länge: Maximal 1 Seite.
    \fi

    In diesem Bericht geht es um die Entwicklung eines Visualisierungstools für MongoDB Datenbanken.
    Ziel dieses Tools ist es, aus den Dokumenten und Collections einer MongoDB Datenbank Schemas zu extrahieren, welche daraufhin anschaulich visualisiert werden.
    Dieses Tool ist Teil eines größeren Datenbank Toolkits, weshalb großer Wert auf Modulatität und Erweiterbarkeit gelegt wird.

    Das MongoDB Visualisierungstool besteht aus einem React Frontend und einem Flask Backend, welche über HTTP miteinander kommunizieren.
    Die Aufgabe des Backends ist es, sich mit einer MongoDB Datenbank zu verbinden.
    Diese Datenbank wir bei erfolgreicher Verbindung analysiert und das Ergebnis der Analyse als JSON zurückgegeben.
    Das Frontend visualisiert diese Daten daraufhin in Form von Tabellen.
\end{abstract}

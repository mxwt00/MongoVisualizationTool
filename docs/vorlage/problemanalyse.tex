\iffalse
Die Analyse des zu lösenden Problems ist Grundlage für jedes 
ingenieurmäßige Vorgehen. Daher soll in diesem Kapitel das zu lösenden 
Problem auf Basis des im Grundlagenkapitel aufbereiteten Wissens 
analysiert werden. Hierzu ist insbesondere notwendig zu klären, wie sich 
das Gesamtproblem in Teilprobleme zerlegen lässt und welche 
Abhängigkeiten zwischen diesen bestehen.

Bei Software-Projekten befindet sich an dieser Stelle typischerweise die 
Anforderungsanalyse des \ac{rup}.

Anforderungen:
\begin{itemize}
    \item modular
    \item erweiterbar
    \item performant
\end{itemize}

\fi



An das MongoDB Visualisierungstool gibt es eine Reihe von Anforderungen, welche erfüllt werden müssen, damit das Projekt gelingen kann:

\section{Anforderungen an das Frontend}
\label{sec:anf_frontend}

Das Resultat des Projekts soll eine Gesamtlösung sein, die aus dem bestehenden ER Modellierungstool und dem MongoDB Visualisierungstool besteht. 
Diese Gesamtlösung soll flexibel durch weitere Anwendungen und Funktionalitäten erweitert werden können.
Deshalb muss das Frontend in einer gemeinsamen Webapp ausgeliefert werden (\nameref{itm:fa1}).
Diese Webapp muss in der Paketstruktur sowie in den verwendeten Komponenten modular sein (\nameref{itm:fa2}).
Um auf die einzelnen Anwendungen der Gesamtlösung zugreifen zu können, wird ein Startbildschirm benötigt, von welchem aus man die Applikationen starten kann (\nameref{itm:fa3}). 

Die Anwendung soll intuitiv benutzbar sein.
Das bedeutet, dass ein Nutzer das System benutzen können muss, ohne vorher eine Benutzeranleitung oder ähnliches zu lesen (\nameref{itm:fa4}).
Dies erfordert auch, dass alle Frontend-Anwendungen der Gesamtlösung sich gleich benutzen lassen und ein einheitliches Design verwenden, da dies sonst den Arbeitsfluss und dadurch die intuitive Bedienung behindert (\nameref{itm:fa5}).

MongoDB Dokumente können beliebig groß und beliebig verschachtelt sein.
Damit die Visualisierung großer Dokumente trotzdem einen Mehrwert hat, müssen diese unabhängig von der Größe übersichtlich und verständlich sein.
Deshalb ist eine übersichtliche Darstellung der Schemata sehr wichtig (\nameref{itm:fa6}).

Die Dokumente in einer Collection müssen nicht das gleiche Schema haben.
Für einen Entwickler kann es hilfreich sein, einen Überblick über alle Variationen in einer Collection zu haben.
Deshalb soll das MongoDB Visualisierungstool für jede Collection eine Detailansicht haben, die diese Variationen visualisiert.
Der unterschied in diesen Variationen kann unter Umständen nicht direkt ersichtlich sein.
Aus diesem Grund müssen diese Abweichungen vom Hauptschema deutlich hervorgehoben werden.(\nameref{itm:fa7})

\section{Anforderungen an das Backend}
\label{sec:anf_backend}

Das MongoDB Visualisierungstool muss auch sehr große MongoDB Datenbanken analysieren können.
Die Analyse sollte jedoch nicht länger als ein paar Sekunden dauern, da dies den Arbeitsfluss der Benutzer unterbrechen würde.
Deshalb müssen die Datenbanken möglichst performant analysiert werden(\nameref{itm:ba1}).

Die Webapp kann potenziell von beliebig vielen Personen gleichzeitig benutzt werden.
Dies bedeutet dass das Backend des MongoDB Visualisierungstools mehrere Anfragen gleichzeitig abarbeiten müssen kann.
Deshalb müssen mehrere MongoDB Datenbanken gleichzeitig verbunden und analysiert werden könnnen(\nameref{itm:ba2}).


\section{Übersicht über die Anforderungen}
\label{sec:anf_uebersicht}

Frontend:

\begin{description}
    \item[\textbf{FA1}\label{itm:fa1}] Gemeinsame Webapp aller Anwendungen der Gesamtlösung
    \item[\textbf{FA2}\label{itm:fa2}] Modularität
    \item[\textbf{FA3}\label{itm:fa3}] Startbildschirm
    \item[\textbf{FA4}\label{itm:fa4}] Intuitive Benutzbarkeit
    \item[\textbf{FA5}\label{itm:fa5}] Einheitliches Design und Layout
    \item[\textbf{FA6}\label{itm:fa6}] Übersichtliche Darstellung der Schemata
    \item[\textbf{FA7}\label{itm:fa7}] Visualisierung der Schema-Variationen in einer Collection
\end{description}

Backend:

\begin{description}
    \item[\textbf{BA1}\label{itm:ba1}] Möglichst performante Analyse
    \item[\textbf{BA2}\label{itm:ba2}] Verbindung und Analyse mehrerer MongoDB Datenbanken gleichzeitig
\end{description}


\iffalse
Die Analyse des zu lösenden Problems ist Grundlage für jedes 
ingenieurmäßige Vorgehen. Daher soll in diesem Kapitel das zu lösenden 
Problem auf Basis des im Grundlagenkapitel aufbereiteten Wissens 
analysiert werden. Hierzu ist insbesondere notwendig zu klären, wie sich 
das Gesamtproblem in Teilprobleme zerlegen lässt und welche 
Abhängigkeiten zwischen diesen bestehen.

Bei Software-Projekten befindet sich an dieser Stelle typischerweise die 
Anforderungsanalyse des \ac{rup}.

Anforderungen:
\begin{itemize}
    \item modular
    \item erweiterbar
    \item performant
\end{itemize}

\fi

An das MongoDB Visualisierungstool gibt es eine Reihe von Anforderungen, welche erfüllt werden müssen, damit das Projekt gelingen kann:

\begin{description}
    \item[\textbf{A1}\label{itm:a1}] Das Visualisierungstool muss das bestehende Er Modellierungstool erweitern.
    \item[\textbf{A2}\label{itm:a2}] Das Gesamtsystem muss weiterhin modular sein, sodass das System ohne Änderungen an den bestehenden Komponenten um weitere Komponenten erweitert werden kann.
    \item[\textbf{A3}\label{itm:a3}] Die Analyse der MongoDB muss performant genug sein, um Datenbanken in wenigen sekunden analysieren zu können.
    \item[\textbf{A4}\label{itm:a4}] Das Frontend muss beliebig aufgebaute MongoDB Dokumente anschaulich visualisieren können.
    \item[\textbf{A5}\label{itm:a5}] Die Anwendung muss von mehreren Nutzern gleichzeitig benutzbar sein.
\end{description}

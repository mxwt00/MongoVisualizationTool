\section{Erreichte Ergebnisse}
\label{sec:ergebnisse}

\iffalse
Die Zusammenfassung dient dazu, die wesentlichen Ergebnisse des 
Praktikums und vor allem die entwickelte Problemlösung und den 
erreichten Fortschritt darzustellen. (Sie haben Ihr Ziel erreicht und 
dies nachgewiesen).
\fi

In diesem Projekt wurde das Problem angegangen, dass es keine geeigneten Schema-Analysetools für MongoDB Datenbanken gibt.
Deshalb wurde das MongoDB Visualisierungstool
Dieses Tool wurde in das bestehende Projekt ER Modellierungstool integriert, um ein erweiterbares Datenbank Toolkit zu bilden.
Das Tool selbst besteht dabei aus 2 Teilen:
Der erste Teil ist ein Frontend, welches in das ER Modellierungstool Frontend integriert wurde.
Dieses Frontend wurde mit JavaScript und React umgesetzt.
Die Modularität dieser Gesamtlösung wurde durch einen Startbildschirm, sowie strukturelle Anpassungen im Code beibehalten und erweitert.
Das MongoDB Visualisierungstool Frontend besteht aus drei Hauptelementen:
\begin{itemize}
    \item Der LeftSidebar, die das Verbinden mit einer MongoDB Datenbank ermöglicht
    \item Dem MongoDiagram, welches einen Überblick über die meistverwendeten Schemas in den Collections gibt
    \item und dem DetailView Popup, welches alle Schemas in einer Collection anzeigt und die Unterschiede hervorhebt.
\end{itemize}

Die Schemas wurden hierbei mit der Komponente DokumentTable umgesetzt, welche auf der Material UI Table Komponente basiert.

Der zweite Teil des MongoDB Visualisierungstools ist ein Backend, welches in dem Python-Framework Flask geschrieben wurde.
Das Backend besitzt einen einzigen REST-Endpunkt.
Die Aufgabe dieses Endpunkts ist es, sich mit der spezifizierten Datenbank zu verbinden und diese zu analysieren.
Die Ergebnisse der Analyse werden an das Frontend im JSON-Format zurückgegeben welches diese dann visualisert.

\section{Ausblick}
\label{sec:ausblick}

\iffalse
Im Ausblick werden Ideen für die Weiterentwicklung der erstellten Lösung 
aufgezeigt. Der Ausblick sollte daher zeigen, dass die Ergebnisse der 
Arbeit nicht nur für die in der Arbeit identifizierten Problemstellungen 
verwendbar sind, sondern darüber hinaus erweitert sowie auf andere 
Probleme übertragen werden können.
\fi

Es gibt ein Paar Funktionen des MongoDB Visualisierungstools, die noch weiter ausgebaut werden könnten.

\subsection{Referenzen}
\label{sub:ausblick_referenzen}

Einerseits Ist das Ermitteln der Referenzen noch nicht besonders genau.
In dem aktuellen Algorithmus wird davon ausgegangen, dass alle Werte des Typs Object ID, welche nicht den namen \_id besitzen, Referenzen sind.
Jedoch kann es auch Referenzen geben, die nicht den Typ Object ID besitzen, da der Primary Key eines Dokuments nicht zwangsläufig eine Object ID sein muss.
Deshalb könnte man die Bestimmung der Referenzen noch weiter verbessern.

Die Visualisierung der Referenzen könnte ebenfalls ausgebaut werden:
Man könnte eine Ansicht der Collections als Graph mit den Collections als Knoten und den Referenzen als Kanten darstellen.
Dies würde es Entwicklern bei Datenbanken mit vielen Referenzen erleichtern, die Abhängigkeiten zwischen den Collections zu Überblicken.

\subsection{Performance der Analyse}
\label{sub:ausblick_performance}

Wie in dem Experiment im Kapitel \nameref{cha:evaluierung} festgestellt wurde, kann die Analyse von Größeren Datenbanken lange dauern.
Die Analyse der größten Datenbank im Experiment dauerte 30 Sekunden.
Solch eine lange Wartezeit unterbricht den Arbeitsablauf eine Entwicklers, und sollte deswegen möglichst minimiert werden,
Dafür gibt es mehrere Ansätze:

Mittels Multithreading die Analysezeit erheblich reduziert werden.
Beispielsweise könnte jede Collection auf einem eigenen Thread analysiiert werden.
Da das Backend in Python geschrieben wurde, ist die Umsetzung von Multithreading relativ simpel.
Python bietet mit der Bibliothek MultiProcessing eine simple, aber effiziente Implementierung von Multithreading.
~\autocite{python:multiprocessing}

Eine weitere Möglichkeit, die Laufzeit der Analyse zu senken, ist, in größeren Datenbanken optional nur einen Teil der Dokumente jeder Collection zu analysieren.
Damit könnte, je nach Anzahl der nicht analysierten Dokumente, die Laufzeit linear gesenkt werden.
Jedoch ist es mit dieser Methode nicht mehr möglich, alle Schemavariationen einer Collection zuverlässig anzuzeigen.

Abgesehen von Software-Seitigen Optimierungen kann die Laufzeit verbessert werden, in dem man das Backend auf einem Leistungsstärkeren Server ausliefert.

\subsection{Weitere Datenbanksysteme}
\label{sub:ausblick_weitere_dbs}

Neben MongoDB gibt es noch eine Vielzahl weiterer Datenbanksysteme, für welche keine geeigneten Visualisierungstools existieren.
Man könnte das MongoDB Visualisierungstool beispielsweise zu einem NoSQL Visualisierungstool erweitern.
Man könnte dafür in der LeftSideBar den Datenbanktyp auswählen, und daraufhin in dem Hauptbildschirm eine passende Visualisierung der Datenbank generieren.
Vor allem für Dokument-Datenbanken und andere zu MongoDB ähnliche Datenbanken wäre der Aufwand, diese zu integrieren, gering, da die nötigen Komponenten zur Visualisierung dieser bereits existieren, und lediglich angepasst werden müssten.
Aber auch andere NoSQL Datenbanken, wie beispielsweise Graph-Datenbanken, könnten in das Visualisierungstool integriert werden.

\subsection{Weitere Tools}
\label{sub:ausblick_tools}

Neben den zwei bestehenden Tools kann das Datenbank Toolkit um weitere Tools erweitert werden.
Das Datenbank Toolkit ist entsprechend modular aufgebaut, dass weitere Tools ohne große Änderungen in das bestehende Toolkit integriert werden können.


\documentclass[oneside]{ausarbeitung}
\bibliography{latexlit}


% ----------------------------------------------------------------------

\begin{document}

%--- Sprachauswahl
% Erlaubte Werte:
%   \selectlanguage{english}
%   \selectlanguage{ngerman}
\selectlanguage{ngerman}

%--- Art der Arbeit
% Erlaubte Werte:
%   \Praxissemesterbericht
%   \Projektbericht
%   \Bachelorarbeit
%   \Seminararbeit
%   \Masterarbeit

\Projektbericht

%--- Studiengang:
% Erlaubte Werte:
%   \Informatik
%   \Elektronik
%   \DataScience
\Informatik

\title{Visualisierung von MongoDB Datenbanken}

\author{Max Winter}
\matrikelnr{80559}

%--- Ist der Erstbetreuer (\examinerA) an der Hochschule ein Professor?
% Erlaubte Werte:
%   \examinerIsAProfessortrue   % Ja
%   \examinerIsAProfessorfalse  % Nein
\examinerIsAProfessortrue   % Ja

%--- Betreuer
\examinerA{Prof.~Dr.~Gregor Grambow}
%\examinerB{Prof.~Dr.~Ulrich~Klauck}

%--- Einreichungsdatum
\date{01. Dezember 2016}

%--- Angaben zur Firma
% Auskommentieren, wenn die Arbeit nicht bei einer ext. Firma gemacht wurde.
%\companyname{Beispielfirma}
%\industrialsector{Beispielbranche}
%\department{Beispielabteilung}
%\companystreet{Beispielstr. 1}
%\companycity{12345 Musterstadt}

%--- Angaben zum Betreuer bei dieser Firma
%\advisorname{Name des Betreuers}
%\advisorphone{(01234) 567-890}
%\advisoremail{name@company.xxx}

%--- Titelseite Anzeigen
\maketitle
\cleardoublepage

%---
\pagenumbering{roman}
\setcounter{page}{1}

%--- Firmendaten Anzeigen
% Auskommentieren, wenn die Arbeit nicht bei einer ext. Firma gemacht wurde.
%\makeworkplace
%\cleardoublepage

%--- Eidesstattliche Erklärung anzeigen
\makeaffirmation
\cleardoublepage

%---
\begin{abstract}

    \iffalse
    Ziel der Kurzfassung ist es, einen (eiligen) Leser zu informieren, so 
    dass dieser entscheiden kann, ob der Bericht für ihn hilfreich ist oder 
    nicht (neudeutsch: Management Summary). Die Kurzfassung gibt daher eine 
    kurze Darstellung

    \begin{itemize}
    \item des in der Arbeit angegangenen Problems
    \item der verwendeten Methode(n)
    \item des in der Arbeit erzielten Fortschritts.
    \end{itemize}

    Dabei sollte nicht auf die Struktur der Arbeit eingegangen werden, also 
    Kapitel~\ref{cha:grundlagen} etc. denn die Kurzfassung soll ja gerade 
    das Wichtigste der Arbeit vermitteln, ohne dass diese gelesen werden muss.
    Eine Kapitelbezogene Darstellung sollte sich in Kapitel~%
    \ref{cha:einleitung} unter Vorgehen befinden.

    Länge: Maximal 1 Seite.
    \fi

    In diesem Bericht geht es um die Entwicklung eines Visualisierungstools für MongoDB Datenbanken.
    Ziel dieses Tools ist es, aus den Dokumenten und Collections einer MongoDB Datenbank Schemas zu extrahieren, welche daraufhin anschaulich visualisiert werden.
    Dieses Tool ist Teil eines größeren Datenbank Toolkits, weshalb in der Entwicklung großen Wert auf Modulatität und Erweiterbarkeit gelegt wird.

    Das MongoDB Visualisierungstool besteht aus einem React Frontend und einem Flask Backend, welche über HTTP miteinander kommunizieren.
    Die Aufgabe des Backends ist es, sich mit einer MongoDB Datenbank zu verbinden.
    Diese Datenbank wird bei erfolgreicher Verbindung analysiert und das Ergebnis der Analyse an das Frontend zurückgegeben.
    Das Frontend visualisiert diese Daten daraufhin in Form von Tabellen.
    
    Das Erarbeiten dieses Tools wird in diesem Bericht von der Problemstellung über die Planung und Entwicklung bis hin zu der Auslieferung erläutert.
    In der Einleitung wird zunächst die Bedeutung von MongoDB und die Wichtigkeit dieses Tools herausgearbeitet.
    Im Kapitel Grundlagen werden daraufhin einige Technologien vorgestellt, die für die Lösung des Problems benötigt werden.
    Welche Anforderungen das Tool erfüllen muss, um als Erfolg gewertet werden zu können, wird im Kapitel Problemanalyse herausgearbeitet.
    Im Lösungskonzept werden zunächst die bestehenden Analyse- und Visualisierungstools für MongoDB analysiert, um den Nutzen und die Besonderheit des MongoDB Visualisierungstools herauszufinden.
    Daraufhin wird ermittelt, mit welchen Tools sich die in der Problemanalyse erarbeiteten Anforderungen am besten erfüllen lassen.
    Anschließend wird geplant, wie die konkrete Umsetzung aussehen soll.
    Im Backend geschiet dies vor allem anhand von UML-Diagrammen, im Frontend anhand von Wireframes.
    In der Implementierung wird die Umsetzung dieses Lösungskonzepts in Python und JavaScript dokumentiert.
    Abschließend wird noch die Auslieferung der Lösung, sowie die Evaluierung der Ergebnisse und einen Ausblick auf die weitere Entwicklung des Database Toolkits präsentiert.
    
    
\end{abstract}

%-----------------------------------------------------------------------
\cleardoublepage
\tableofcontents

%---
\listoffigures

%---
\listoftables

%---
\lstlistoflistings

%---
\listofabbreviations
\begin{acronym}[Bsp.]  % Längstes Kürzel in der nachfolgenden
                       % Liste um die Breite der Spalte für die
                       % Abkürzungen zu bestimmen.

%% Eintrag: \acro{Referenzname}[Kürzel]{Langform}
%% Im Text wird die Abkürzung dann mit \ac{Referenzname} benutzt.
\acro{rup}[RUP]{Rational Unified Process}
\acro{sql}[SQL]{Structured Query Language}
\acro{http}[HTTP]{Hypertext Transfer Protocol}
\acro{api}[API]{Application Programming Interface}
\acro{rest}[REST]{Representational State Transfer}
\acro{json}[JSON]{JavaScript Object Notation}
\acro{url}[URL]{Uniform Resource Locator}
\acro{ui}[UI]{User Interface}   
\acro{uri}[URI]{Uniform Resource Identifier}
\acro{uuid}[UUID]{Universally Unique Identifier}
\acro{dto}[DTO]{Data Transfer Object}
\acro{ddl}[DDL]{Data Definition Language}
\acro{dml}[DML]{Data Manipulation Language}
\acro{xml}[XML]{Extensible Markup Language}
\acro{jaxrs}[JAX-RS]{Jakarta RESTful Web Services}
\acro{pojo}[POJO]{Plain Old Java Object}
\end{acronym}
%---


\cleardoublepage
\pagenumbering{arabic}
\setcounter{page}{1}

% ----------------------------------------------------------------------
\chapter{Einleitung}
\label{cha:einleitung}
\iffalse
Die Einleitung dient dazu, beim Leser Interesse für die Inhalte 
Praxissemesterberichts zu wecken, die behandelten Probleme aufzuzeigen 
und die zu ihrer Lösung entwickelten Konzepte zu beschreiben.
\fi

\section{Motivation}
\label{sec:motivation}

\iffalse
In der Motivation wird dargestellt, welche Bedeutung die im 
Praxissemester zu entwickelnden Lösungen für das betreuende Unternehmen 
haben. Es wird beispielsweise aufzeigt, in welches Produkt sie eingehen, 
welcher Ablauf verbessert werden soll etc.
\fi

Für einen Entwickler, der mit einer Datenbank arbeitet, ist es wichtig zu wissen, wie diese Datenbank aussieht, was für Abhängigkeiten es gibt und ob die Implementierung auch wirklich der Planung entspricht.
Aus diesem Grund werden Visualisierungs- und Analysetools für Datenbanken benötigt.
MongoDB hat sich in den letzten Jahren zu einem der wichtigsten Datenbanksysteme entwickelt, da aufgrund immer größerer werdenden Datenmengen die Vorteile von NoSQL-Datenbanken für immer mehr Anwendungen überwiegen.
~\autocite{db-engines:mongodb}
Da MongoDB keine relationale Datenbank ist, können herkömmliche Visualisierungs- und Analysetools, die für SQL entwickelt wurden, nicht verwendet werden.
Für MongoDB gibt es zwar Visualisierungstools, wie Beispielsweise MongoDB Charts und MongoDB Compass, die meisten davon visualisieren aber die Daten in der Datenbank und nicht die Struktur und das Schema der Dokumente in der Datenbank.
~\autocite{knowi:mongo_vis_tools}
Die Datenmenge kann jedoch unter Umständen sehr groß sein, was ein vollständiges Überblicken der Datenbank schwierig bis unmöglich macht.
Auß diesem Grund soll das Ziel dieser Arbeit sein, solch ein Visualisierungstool zu entwickeln.

\section{Problemstellung und -abgrenzung}
\label{sec:problemstellung}

\iffalse
Die Problemstellung dient dazu, das zu lösende Problem klar zu 
definieren und abzugrenzen. Der Praktikant soll ein klares Verständnis 
des zu lösenden Problems haben. Insbesondere soll auch verhindert 
werden, dass zu viele Probleme gleichzeitig angegangen werden. Eine 
Negativabgrenzung verhindert, dass beim Leser später nicht erfüllte 
Erwartungen geweckt werden.
\fi

Es soll ein MongoDB Visualisierungstool entwickelt werden, welches eine MongoDB Datenbank analysiert und auswertet.
Daraus sollen Schemas extrahiert und visualisiert werden.
Das Visualisieren der konkreten Daten in der MongoDB Datenbank ist nicht Teil des Problems.

\section{Vorangegangene Arbeit}
\label{sec:vorangegangene_arbeit}

Das MongoDB Visualisierungstool soll in eine vorangegangene Arbeit integriert werden.
In dieser vorangegangenen Arbeit wurde ein Entity-Relationship Modellierungstool entwickelt.
Eines der Ziele dieses Modellierungstools war es, die Anwendung möglichst modular zu gestalten, damit das Modellierungstool langfristig zu einem umfassenden Datenbank-Toolkit erweitert werden kann.
~\autocite{ruttmann:projektarbeit}
Ein Hauptfokus des MongoDB Visualisierungstools liegt deshalb darauf, diese Modularität beizubehalten und gegebenenfalls weiter zu verbessern.

\section{Ziel der Arbeit}
\label{sec:ziel}

Ziel dieses Projekts ist es, ein Visualisierungstool für MongoDB Datenbanken zu entwickeln, welches die Dokumente einer Datenbank analysiert und auswertet.
Daraus sollen Schemas abgeleitet werden, welche anschließend visualisiert werden.
Dies erleichtert es Entwicklern von Datenbanken, die Strukturen und Abhängigkeiten in ihren Datenbanken zu verstehen und zu verbessern.
Dafür soll eine Backendanwendung zur Verbindung und Analyse von MongoDb Datenbanken entwickelt werden, sowie ein Webfrontend für die Visualisierung der analysierten Daten.
Das Visualisierungstool soll dabei in die vorangegangene Arbeit integriert werden und die Gesamtlösung dabei modular gehalten werden.

Um Datenbanken analysieren zu können, muss es möglich sein, sich mit diesen zu verbinden.
Dies erfordert einerseits eine Benutzeroberfläche, über die der Nutzer die Verbindungsdaten eingeben kann.
Andererseits erfordert dies die Verbindung mit der Datenbank selbst über eine geeignete MongoDB Schnittstelle im Backend.
Die Dokumente der Datenbank müssen anschließend analysiert werden, um aus ihnen Schemas zu extrahieren.
Diese Schemas müssen daraufhin übersichtlich und visuell ansprechend angezeigt werden.

\iffalse
\section{Vorgehen}
\label{sec:vorgehen}

Nachdem mit Problemstellung und Ziel gewissermaßen Anfangs- und Endpunkt 
des Praktikums beschrieben sind, wird hier der zur Erreichung des Ziels 
eingeschlagene Weg vorgestellt. Dazu werden typischerweise die folgenden 
Kapitel und ihr Beitrag zur Erreichung des Ziels der Arbeit kurz 
beschrieben. Die folgenden Kapitel sind ein – möglicher – Aufbau, 
Abweichungen können durchaus notwendig sein. Zur Darstellung des 
Vorgehens ist eine grafische Darstellung sinnvoll, bei der die einzelnen 
Lösungsschritte und ihr Zusammenhang dargestellt werden. Ein Beispiel 
hierfür findet sich in Abbildung \ref{fig:1}.
\fi


% ---
\chapter{Grundlagen}
\label{cha:grundlagen}
\if False
In diesem Kapitel das für das Praktikum relevante Grundlagenwissen 
dargestellt. Der Praktikant soll hierzu das ihm durch Vorlesungen 
bekannte, bzw. durch Recherchen vertiefte theoretische Wissen 
darstellen, das für die Lösung der im Praktikum gestellten Probleme 
notwendig ist.

Dabei ist darauf zu achten, nur solche Inhalte in das Grundlagenkapitel 
aufzunehmen, die später auch verwendet werden (Problembezogenheit). 
Ebenso ist auf eine ausreichend tiefe und vollständige Darstellung der 
Grundlagen zu achten.

Für die Erstellung des Literaturverzeichnisses 
wird das Werkzeug JabRef\autocite{JabRef:JabRef} verwendet. 

Sie können aber auch das Werkzeug Citavi\autocite{SAS:Citavi} benutzen
und dort nach \textsc{Bib}\TeX{} exportieren.
\fi

\section{JSON}
\label{sec:json}

\ac{json} ist ein Format zum Datenaustausch.
\ac{json} basiert auf der Syntax von JavaScript Objekten, ist jedoch unabhängig von der Programmiersprache einsetzbar.
Vorteile von \ac{json} sind die einfache Lesbarkeit für Menschen und das einfache parsen und generieren für Maschinen.
~\autocite{json:json}

\begin{figure}[H]
    \begin{minipage}[t]{0.45\textwidth}
        \flushleft\textbf{Object} ist ein Set aus Key/Value Paaren.
        \includegraphics[width=0.9\textwidth]{images/json_object}
        \caption{JSON Object}
        \label{fig:json_object}
    \end{minipage}\hfill
    \begin{minipage}[t]{0.45\textwidth}
        \flushleft\textbf{Array} ist eine geordnete Liste von Values.
        \includegraphics[width=0.9\textwidth]{images/json_array}
        \caption{JSON Array}
        \label{fig:json_array}
    \end{minipage}\hfill
\end{figure}
\begin{figure}[H]
    \begin{minipage}[t]{0.45\textwidth}
        \textbf{Value} kann ein String, eine Zahl, ein Boolean, ein Objekt, ein Array oder null sein.
        Dabei können beliebig viele Values ineinander verschachtelt sein.
        \includegraphics[width=0.9\textwidth]{images/json_value}
        \caption{JSON Value}
        \label{fig:json_value}
    \end{minipage}\hfill
    \begin{minipage}[t]{0.45\textwidth}
        \flushleft\textbf{String} ist eine Sequenz aus Unicode Buchstaben.
        \includegraphics[width=0.9\textwidth]{images/json_string}
        \caption{JSON String}
        \label{fig:json_string}
    \end{minipage}\hfill
\end{figure}
\begin{figure}[H]
    \begin{minipage}[t]{0.45\textwidth}
        \textbf{Number} ist eine Zahl.
        Number kann sowohl eine Gleitkommazahl als auch eine Ganzzahl sein, und kann positiv sowie negativ sein.
        \includegraphics[width=0.9\textwidth]{images/json_number}
        \caption{JSON Number}
        \label{fig:json_number}
    \end{minipage}\hfill
\end{figure}

\section{HTTP}
\label{sec:http}

\ac{http} ist ein Protokoll, das benutzt wird, um über das Internet zu kommunizieren.
Hauptsächlich wird \ac{http} für die Kommunikation zwischen einem Webbrowser und einem Webserver benutzt.
In \ac{http} wird mittels Nachrichten kommuniziert.
Es wird erst ein Request vom Client abgesetzt, der daraufhin von dem Server mit einer Response beantwortet wird.
Diese Nachrichten bestehen aus einem Header und einem Body.
Der Body enthält die Nachricht selbst.
Der Header enthält Meta-Daten über einen Request, wie beispielsweise die angefragten Ressourcen und Datentypen des Body.

\section{SQL}
\label{sec:sql}

Die \ac{sql} ist eine Datenbanksprache für relationale Datenbanken.
In \ac{sql} werden die Daten in Tabellen organisiert, die ein festes Schema definieren.
Man kann \ac{sql} in 2 Teile aufteilen:
Die \ac{ddl} definiert den Aufbau des Schemas.
Mit ihr können Datenbankobjekte erzeugt und gelöscht werden.
Zur \ac{ddl} gehören unter anderem die Befehle CREATE, ALTER, DROP und TRUNCATE\@.
Mit der \ac{dml} können Daten manipuliert, also eingefügt, geändert, gelesen und gelöscht werden.
Die Befehle hierfür lauten SELECT, INSERT, UPDATE, DELETE, MERGE und noch weitere.
~\autocite{schicker:datenbanken}

\subsection{Normalformen}
\label{sub:normalformen}

\section{MongoDB}
\label{sec:mongodb}

MongoDB ist eine Dokument-Orientierte Datenbank. 
In Dokument-Orientierten Datenbanken wird das aus SQL bekannte Konzept von Reihen durch Dokumente ersetzt.
Dokumente haben im Gegensatz zu Reihen in Tabellen kein fixes Schema, dass sie erfüllen müssen, sondern sind sehr flexibel.
Grundsätzlich bestehen Dokumente aus Key - Value Paaren, welche in einer JSON-Ähnlichen Struktur gespeichert werden.
Dokumente können, wie auch in JSON, ineinander verschachtelt sein, was Hierarchische Strukturen und dadurch Denormalisierung ermöglicht.
Die Dokumente werden in Collections Organisiert.
Eine Collection entspricht in SQL einer Tabelle ohne das fixe Schema.
Diese Collections befinden sich wiederum in Datenbanken.
Eine MongoDB Instanz kann mehrere voneinander unabhängige Datenbanken beinhalten.
Man kann sich mit der MongoShell mit einer MongoDB Instanz verbinden um mit der Mongo Query Language oder mit JavaScript die Instanz administrieren und Daten manipulieren.
~\autocite{bradshaw:mongodb}

\section{REST API}
\label{sec:rest}

Ein \ac{api} ist eine Schnittstelle, über die von außen mit einem Programm interagiert werden kann.
Das Grundprinzip von \ac{rest} ist ein Backend, welches Resourcen beinhaltet.
Ein Client kann über die gängigen \nameref{sec:http} Operationen mit diesen Resourcen interagieren.
Jede Resource hat eine global eindeutige ID und wird meist durch \ac{json} oder \ac{xml} repräsentiert.
In Java wird ein \ac{rest}ful Webservice für gewöhnlich mit \ac{jaxrs} umgesetzt.
~\autocite{schiesser:javaEE7}

\section{Python}
\label{sec:python}

Python ist eine Objektorientierte High-Level Programmiersprache, die besonderen Wert auf Lesbarkeit legt.
Variablen in Python werden dynamisch getyped.
Das bedeutet, dass eine Variable in Python keinen festen Typ hat, sondern dieser dynamisch über den ihr zugewiesenen Wert bestimmt wird.
Python ist keine Compilierte, sondern eine interpretierte Programmiersprache.
Diese Eigenschaften machen Python zu der idealen Sprache für das Schreiben von Skripten, sowie für Anwendungen, die schnell und simpel entwickelt werden sollen.
PIP ist ein in Python integrierter Paketmanager, der das installieren von Packages erleichtert.
Der Python Interpreter ist in C geschrieben, weshalb man die Funktionalität des Interpreters durch C-Programme erweitern kann, um beispielsweise laufzeitkritische Funktionen in C auszuführen.
~\autocite{van:python}

\section{Flask}
\label{sec:flask}

Flask ist ein minimalistisches Open-Source Web Framework.
Flask kommt mit ein paar Kernpaketen, welche für Web Apps benötigt werden.
Alles weitere muss der Nutzer selbst über PIP installieren:
Durch diesen minimalistischen Ansatz kann man Flask sehr flexibel einsetzen.
Flask ist beispielsweise flexibel mit SQL oder NoSQL Datenbanken, aber auch ohne Datenbanken einsetzbar.
~\autocite{grindberg:flask}

% TODO DOM

\section{JavaScript}
\label{sec:js}

JavaScript ist eine High-Level Programmiersprache, die just-in-time kompiliert wird.
JavaScript ist besonders bekannt als Skriptsprache für Webseiten.
Die Pakete in JavaScript werden mittels dem Paketmanager npm verwaltet.
% TODO mehr zu JavaScript schreiben, Quelle fehlt

\section{React}
\label{sec:react}

React ist eine JavaScript Bibliothek zum bauen Von Benutzeroberflächen.
React ist Komponentenbasiert.
Das bedeutet, das React aus einzelnen Komponenten bestehen, die ihren eigenen State haben, und die zusammengesetzt ein komplexes \ac{ui} bilden.
~\autocite{banks:react}


\subsection{Redux Store}
\label{sub:redux}

Redux ist ein State Container für JacaScript Apps, der es ermöglicht, States nicht mehr in den Komponenten, sondern in einem Zentralen Container zu speichern.
Dadurch können beispielsweise States wiederhergestellt werden, nachdem eine Komponente geschlossen und wieder geöffnet wurde.
Außerdem erleichtert Redux es, State Änderungen in Komponenten nachzuvollziehen.
~\autocite{freecodecamp:redux}

\subsection{Axios}
\label{sub:axios}

\subsection{Material UI}
\label{sub:mui}


%---
\chapter{Problemanalyse}
\label{cha:problemanalyse}
\iffalse
Die Analyse des zu lösenden Problems ist Grundlage für jedes 
ingenieurmäßige Vorgehen. Daher soll in diesem Kapitel das zu lösenden 
Problem auf Basis des im Grundlagenkapitel aufbereiteten Wissens 
analysiert werden. Hierzu ist insbesondere notwendig zu klären, wie sich 
das Gesamtproblem in Teilprobleme zerlegen lässt und welche 
Abhängigkeiten zwischen diesen bestehen.

Bei Software-Projekten befindet sich an dieser Stelle typischerweise die 
Anforderungsanalyse des \ac{rup}.

Anforderungen:
\begin{itemize}
    \item modular
    \item erweiterbar
    \item performant
\end{itemize}

\fi

An das MongoDB Visualisierungstool gibt es eine Reihe von Anforderungen, welche erfüllt werden müssen, damit das Projekt gelingen kann:

\begin{description}
    \item[\textbf{A1}\label{itm:a1}] Das Visualisierungstool muss das bestehende Er Modellierungstool erweitern.
    \item[\textbf{A2}\label{itm:a2}] Das Gesamtsystem muss weiterhin modular sein, sodass das System ohne Änderungen an den bestehenden Komponenten um weitere Komponenten erweitert werden kann.
    \item[\textbf{A3}\label{itm:a3}] Die Analyse der MongoDB muss performant genug sein, um Datenbanken in wenigen sekunden analysieren zu können.
    \item[\textbf{A4}\label{itm:a4}] Das Frontend muss beliebig aufgebaute MongoDB Dokumente anschaulich visualisieren können.
    \item[\textbf{A5}\label{itm:a5}] Die Anwendung muss von mehreren Nutzern gleichzeitig benutzbar sein.
\end{description}


%---
\chapter{Lösungskonzept}
\label{cha:loesungskonzept}
\iffalse
Auf der Basis der im vorangegangenen Kapitel erstellten Problemanalyse 
und der im Grundlagenkapitel aufgearbeiteten theoretischen Kenntnisse 
wird ein Lösungskonzept erarbeitet.

Bei Software-Projekten entspricht dieses Kapitel typischerweise der 
Analyse \& Design-Phase des \ac{rup}. Typische Ergebnisse dieser Phase sind 
Klassendiagramme etc.
\fi

\section{Verwendete Technologien}
\label{sec:verwendete_technologien}

\begin{itemize}
    \item Wahl des Backend Frameworks  und der Sprache
    \begin{itemize}
        \item Probleme mit MongoDB Client in Java/Spring
        \item Python und Flask sind lightweight
    \end{itemize}
    \item Wahl des Frontend Frameworks und der Sprache
    \begin{itemize}
        \item bestehendes Projekt mit diesem Framework
        \item Verbreitung von React
        \item Modularität dank React
        \item Warum Web?
    \end{itemize}
\end{itemize}

Da die Analyse der MongoDB Dokumente sehr rechenintensiv ist, wird die Analyse in ein Backend ausgelagert.
Um den zuvor definierten Anforderungen gerecht zu werden, ist es wichtig, ein geeignetes Backend-Framwork auszuwählen.
Das Er Modellierungstool nutzt Java Spring als Backend-Framework.
Da man zum Teil Code von dem bestehenden Backend übernehmen könnte, bietet es sich deshalb an, in diesem Projekt ebenfalls Spring zu verwenden.
Für Spring gibt es eine MongoDB Implementierung namens Spring Data MongoDB.
Diese Implementierung ist jedoch dafür ausgelegt, \ac{pojo}s auf Dokumente zu mappen.
Im MongoDB Visualisierungstool sollen hingegen MongoDB Dokumente dynamisch eingelesen und analysiert werden.
Um ~\nameref{itm:a5} zu erfüllen, ist es desweiteren nötig, beliebig viele verschiedene Datenbanken gleichzeitig zu verbinden und zu analysieren.
Die Verbindung mit MongoDB Datenbanken Spring Data MongoDB erfolgt jedoch mit festgelegten Datenbanken, welche in der application.properties Datei definiert werden.
~\autocite{spring:spring-data-mongodb}
Deshalb ist die Spring Data MongoDB Bibliothek für diese Anwendung nicht geeignet.
Neben der Spring Data MongoDB Bibliothek gibt es auch noch einen anderen MognoDB Java Client, Java Sync.
Dieser funktioniert jedoch nicht zusammen mit dem Spring Framework.
Aus diesem Grund kann Spring sowie andere Java Backend Frameworks nicht genutzt werden.

Als alternatives Backend Framework mit REST API dazu bot sich Flask an.
Ein großer Vorteil von Flask ist, dass Flask sehr minimal ist und nur mit dem minimum an benötigten Bibliotheken vorkonfiguriert ist.
Spring ist im Gegensatz dazu ein sehr mächtiges Framework mit vielen Features, von denen in diesem Projekt aber nur sehr wenige gebraucht werden.
Ein weiterer Vorteil von Flask sowie von Python ist die Schlankheit des Codes.
In Python lässt sich meist die gleiche Funktionalität in weniger Code schreiben als in Java.
Dazu kommt, dass in Flask sehr viel weniger Boilerplate Code benötigt wird als in Spring.
Ein minimaler Endpunkt in Flask lässt sich bereits mit 2 Zeilen Code umsetzen.
Jedoch hat Flask nicht nur Vorteile gegenüber Spring:
Flask ist grundsätzlich deutlich unperformanter als Spring.
Dies liegt unter anderem daran, dass Python eine interpretierte Sprache ist, und Java eine kompilierte.
~\autocite{sverker:rest_comparison}
Dies widerspricht zunächst der Anforderung ~\nameref{itm:a5}.
Die Performance-Probleme lassen sich aber durch Multiprocessing ausgleichen.
Multiprocessing bedeutet, dass bestimmte Teile der Berechnung auf mehrere Threads im Prozessor aufgeteilt werden und dadurch parallell ausgeführt werden.
Python bietet eine simpel zu implementierende Lösung für Multiprocessing an, welche man bei der Analyse der Dokumente der MongoDB Datenbanken gut einsetzen kann.
Beispielsweise kann die Analyse jeder Collection von einem extra Thread ausgeführt werden.




\section{Bestehende Visualisierungstools}
\label{sec:bestehende_visualisierungstools}

\section{Analyse der MongoDB Datenbank}
\label{sec:mongoDB_analyse}

\section{Planung des Frontends}
\label{sec:planung_frontend}


%---
\chapter{Implementierung}
\label{cha:implementierung}
In diesem Kapitel wird die konkrete Implementierung des im Kapitel
\ref{cha:loesungskonzept} entwickelten Lösungskonzepts beschrieben.
Hierbei wird auf die konkret verwendeten Entwicklungswerkzeuge etc. 
Bezug genommen.

Bei Software-Projekten besteht dieses Kapitel typischerweise aus den 
Phasen Implementierung \& Test im \ac{rup}.

Zum Beispiel kann man hier auch ein kleines Listing einfügen.

\begin{lstlisting}[language=c,%
                   caption={Überschrift des Quelltexts}]
#include<stdio.h>

int main() {
    // Kommentar
    int answer = 20 << 1;
    answer += 2;
    printf("Hallöchen Welt!\n");
    printf("Die Antwort ist: %d\n", answer);
    return 0;
}
\end{lstlisting}

Manchmal hilft auch eine kleine Tabelle:

\begin{table}[htbp]
\centering
\begin{tabular}{|l|r|}
\hline
\textbf{Messwert a} & \textbf{Messwert b} \\ \hline
9 & 5 \\ \hline
1 & 4 \\ \hline
1 & 3 \\ \hline
\end{tabular}
\caption{Überschrift der Tabelle}
\label{tab:my-table}
\end{table}

Details siehe Tabelle~\ref{tab:my-table}.

\section{Backend}
\label{sec:backend}

\subsection{Analyse der MongoDB Datenbank}
\label{sub:mongoDB_analyse}

\subsection{REST API}
\label{sub:rest_api}

\section{Frontend}
\label{sec:frontend}

\subsection{}
\label{sub:rest_api}



%---
\chapter{Inbetriebnahme}
\label{cha:inbetriebnahme}
\iffalse
Aufgabe des Kapitels Inbetriebnahme ist es, die Überführung der in 
Kapitel \ref{cha:implementierung} entwickelte Lösung in das betriebliche 
Umfeld aufzuzeigen. Dabei wird beispielsweise die Inbetriebnahme eines 
Programms beschrieben oder die Integration eines erstellten 
Programmodules dargestellt.

Bei der Software-Erstellung entspricht dieses Kapitel der 
Auslieferungsphase (Deployment) im \ac{rup}.
\fi

Das MongoDB Visualisation Tool wird auf einem Linux Server mit Debian 11 gehostet.
Das Flask Backend läuft in einem Docker Container.
Der Vorteil von Docker ist, dass der Applikation eine eigene Umgebung bereitgestellt wird, die unabhängig vom Host-System ist.
Um das Backend als Dockercontainer auszuliefern, wurde das vorkonfigurierte Docker Image uwsgi-nginx-flask von tiangolo genutzt.
In diesem Docker Image ist uWSGI und Nginx vorinstalliert, was das ausliefern von Flask Applikationen erleichtert.
~\autocite{tiangolo:uwsgi-nginx-flask}

Das Backend für das ER Modelling Tool ist als Heroku-App deployed.
Heroku ist ein cloudbasierter Serviceanbieter, welcher ein kostenloses Modell für das Hosting von Software und DNS anbietet.

Das Frontend wurde mit NPM gebaut und auf dem Server bereitgestellt.
Mittels NGINX wurde das Webinterface daraufhin gehostet. 
Das Interface ist nun unter der Adresse \url{http://kodewisch.com:12030} erreichbar.


%---
\chapter{Evaluierung}
\label{cha:evaluierung}

Aufgabe des Kapitels Evaluierung ist es, in wie weit die Ziele der 
Arbeit erreicht wurden. Es sollen also die erreichten Arbeitsergebnisse 
mit den Zielen verglichen werden. Ergebnis der Evaluierung kann auch 
sein, das bestimmte Ziele nicht erreicht werden konnten, wobei die 
Ursachen hierfür auch außerhalb des Verantwortungsbereichs des 
Praktikanten liegen können.

%---
\chapter{Zusammenfassung und Ausblick}
\label{cha:zusammenfassung}
\section{Erreichte Ergebnisse}
\label{sec:ergebnisse}

\iffalse
Die Zusammenfassung dient dazu, die wesentlichen Ergebnisse des 
Praktikums und vor allem die entwickelte Problemlösung und den 
erreichten Fortschritt darzustellen. (Sie haben Ihr Ziel erreicht und 
dies nachgewiesen).
\fi

In diesem Projekt wurde das Problem angegangen, dass es keine geeigneten Schema-Analysetools für MongoDB Datenbanken gibt.
Deshalb wurde das MongoDB Visualisierungstool entwickelt.
Dieses Tool wurde in das bestehende Projekt ER Modellierungstool integriert, um ein erweiterbares Datenbank Toolkit zu bilden.
Das Tool selbst besteht dabei aus 2 Teilen:
Der erste Teil ist ein Frontend, welches in das ER Modellierungstool Frontend integriert wurde.
Dieses Frontend wurde mit JavaScript und React umgesetzt.
Die Modularität dieser Gesamtlösung wurde durch einen Startbildschirm, sowie strukturelle Anpassungen im Code erweitert.
Das MongoDB Visualisierungstool Frontend besteht aus drei Hauptelementen:
\begin{itemize}
    \item Der LeftSidebar, die das Verbinden mit einer MongoDB Datenbank ermöglicht
    \item Dem MongoDiagram, welches einen Überblick über die meistverwendeten Schemas in den Collections gibt
    \item und dem DetailView Popup, welches alle Schemas in einer Collection anzeigt und die Unterschiede hervorhebt.
\end{itemize}

Die Schemas wurden hierbei mit der Komponente DokumentTable umgesetzt, welche auf der Material UI Table Komponente basiert.

Der zweite Teil des MongoDB Visualisierungstools ist ein Backend, welches in dem Python-Framework Flask geschrieben wurde.
Das Backend besitzt einen einzigen REST-Endpunkt.
Die Aufgabe dieses Endpunkts ist es, sich mit der spezifizierten Datenbank zu verbinden und diese zu analysieren.
Die Ergebnisse der Analyse werden an das Frontend im JSON-Format zurückgegeben, welches diese anschließend visualisert.

\section{Ausblick}
\label{sec:ausblick}

\iffalse
Im Ausblick werden Ideen für die Weiterentwicklung der erstellten Lösung 
aufgezeigt. Der Ausblick sollte daher zeigen, dass die Ergebnisse der 
Arbeit nicht nur für die in der Arbeit identifizierten Problemstellungen 
verwendbar sind, sondern darüber hinaus erweitert sowie auf andere 
Probleme übertragen werden können.
\fi

Es gibt ein Paar Funktionen des MongoDB Visualisierungstools, die noch weiter ausgebaut werden könnten.

\subsection{Referenzen}
\label{sub:ausblick_referenzen}

Einerseits erfolgt das Ermitteln der Referenzen noch nicht sehr präzise.
In dem aktuellen Algorithmus wird davon ausgegangen, dass alle Werte des Typs Object ID, welche nicht den Namen \_id besitzen, Referenzen sind.
Jedoch kann es auch Referenzen geben, die nicht den Typ Object ID besitzen, da der Primary Key eines Dokuments nicht zwangsläufig eine Object ID sein muss.
Deshalb könnte man die Bestimmung der Referenzen noch weiter verbessern.

Die Visualisierung der Referenzen könnte ebenfalls ausgebaut werden:
Man könnte eine Ansicht der Collections als Graph mit den Collections als Knoten und den Referenzen als Kanten darstellen.
Dies würde es Entwicklern bei Datenbanken mit vielen Referenzen erleichtern, die Abhängigkeiten zwischen den Collections zu überblicken.

\subsection{Performance der Analyse}
\label{sub:ausblick_performance}

Wie in dem Experiment im Kapitel \nameref{cha:evaluierung} festgestellt wurde, kann die Analyse von großen Datenbanken lange dauern.
Die Analyse der größten Datenbank im Experiment dauerte 30 Sekunden.
Solch eine lange Wartezeit unterbricht den Arbeitsablauf eines Entwicklers und sollte deswegen möglichst vermieden werden.
Dafür gibt es mehrere Ansätze:

Mittels Multithreading kann die Analysezeit erheblich reduziert werden.
Beispielsweise kann jede Collection auf einem eigenen Thread analysiert werden.
Da das Backend in Python geschrieben ist, ist die Umsetzung von Multithreading relativ simpel.
Python bietet mit der Bibliothek MultiProcessing eine simple, aber effiziente Implementierung von Multithreading an.
~\autocite{python:multiprocessing}

Eine weitere Möglichkeit, die Laufzeit der Analyse zu senken, ist, in größeren Datenbanken optional nur einen Teil der Dokumente jeder Collection zu analysieren.
Damit kann die Laufzeit linear gesenkt werden, abhängig von der Anzahl der nicht Dokumente, die nicht analysiert werden.
Jedoch ist es mit dieser Methode nicht mehr möglich, alle Schemavariationen einer Collection zuverlässig anzuzeigen.

Abgesehen von Software-Seitigen Optimierungen kann die Laufzeit verbessert werden, in dem man das Backend auf einem leistungsstärkeren Server ausliefert.

\subsection{Weitere Datenbanksysteme}
\label{sub:ausblick_weitere_dbs}

Neben MongoDB gibt es noch eine Vielzahl weiterer Datenbanksysteme, für welche keine geeigneten Visualisierungstools existieren.
Man kann das MongoDB Visualisierungstool beispielsweise zu einem NoSQL Visualisierungstool erweitern.
Vorstellbar wäre eine Einstellung in der LeftSideBar, in der man den Typ der zu analysierenden Datenbank auswählen kann.
Mit dieser Information kann man daraufhin in der MongoDiagram Komponente eine passende Visualisierung anzeigen.
Vor allem für Dokument-Datenbanken und andere zu MongoDB ähnliche Datenbanken wäre der Aufwand, diese zu integrieren, gering, da die nötigen Komponenten zur Visualisierung dieser bereits existieren und lediglich angepasst werden müssten.
Aber auch andere NoSQL Datenbanken, wie beispielsweise Graph-Datenbanken, könnten in das Visualisierungstool integriert werden.

\subsection{Weitere Tools}
\label{sub:ausblick_tools}

Neben den zwei bestehenden Tools kann das Datenbank Toolkit um weitere Tools erweitert werden.
Das Datenbank Toolkit ist entsprechend modular aufgebaut, sodass weitere Tools ohne große Änderungen in das bestehende Toolkit integriert werden können.


%-----------------------------------------------------------------------
\appendix

%---
\printbibliography[heading=bibintoc]

%---
\chapter{Anhang A}

%---
\chapter{Anhang B}


\end{document}

\iffalse
Aufgabe des Kapitels Evaluierung ist es, in wie weit die Ziele der 
Arbeit erreicht wurden. Es sollen also die erreichten Arbeitsergebnisse 
mit den Zielen verglichen werden. Ergebnis der Evaluierung kann auch 
sein, das bestimmte Ziele nicht erreicht werden konnten, wobei die 
Ursachen hierfür auch außerhalb des Verantwortungsbereichs des 
Praktikanten liegen können.
\fi

Die Anforderung \nameref{itm:fa1} verlangt, dass das MongoDB Visualisation Tool in einer gemeinsamen Webapp mit dem ER Modelling Tool implementiert wird.
Dies wurde durch das Einfügen eines Startbildschirms zur Auswahl des Tools erreicht (\nameref{itm:fa3}).
Der Startbildschirm erleichtert zudem das Einfügen weiterer Tools in das Database Toolkit.
\nameref{itm:fa2} verlangt, dass die Anwendung modular ist, um einfach weitere Tools hinzuzufügen.
Diese Anforderung wird neben dem Startbildschirm durch die verbesserte Package Struktur im Frontend, sowie der Microservice Architektur im Backend erfüllt.

\nameref{itm:fa4} verlangt, dass das Frontend intuitiv Benutzbar ist.
Die Interaktionsmöglichkeiten mit dem MongoDB Visualisation Tool sind relativ klein, da die Hauptaufgabe der Anwendung Visualisierung ist.
Der Großteil der Interaktionsmöglichkeiten befindet sich in der LeftSideBar.
In dieser werden nur Standardkomponenten von Material UI verwendet, deren Benutzung jedem Nutzer geläufig sind.
Zudem sind alle Elemente aussagekräftig beschriftet.
Deshalb ist die Anforderung \nameref{itm:fa4} erfüllt.

Um die Intuitive Benutzbarkeit zu gewährleisten, müssen alle Tools ein einheitliches Design und Layout besitzen (\nameref{itm:fa5}).
Das einheitliche Layout wird durch Verwendung der gleichen Grundelemente wie das ER Modelling Tool gewährleistet.
Um das einheitliche Design sicherzustellenbn, wurden die gleichen Farben, Schriftgrößen und Co. in den CSS Dateien verwendet.
Um Entwicklern weiterer Tools das einhalten eines einheitlichen Designs zu erleichtern, wurde ein Material UI Theme erstellt, und wo möglich, Material UI Komponenten verwendet.

Die übersichtliche Darstellung der Schemata (\nameref{itm:fa6}) wurde durch kustomisierte Material UI Tabellen Realisert.
Durch das dynamische Ausklappen von verschachtelten Dokumenten sowie Arrays lassen sich beliebig tief verschachtelte Schemata übersichtlich darstellen.
Damit die Unterschiede aller Schemata in einer Collection übersichtlich dargestellt werden, gibt es eine Detail View, in der alle Abweichungen des Hauptschemas Farblich markiert sind.
Dadurch ist die Anforderung \nameref{itm:fa7} erfüllt.

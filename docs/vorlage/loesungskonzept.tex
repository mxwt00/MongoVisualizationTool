Auf der Basis der im vorangegangenen Kapitel erstellten Problemanalyse 
und der im Grundlagenkapitel aufgearbeiteten theoretischen Kenntnisse 
wird ein Lösungskonzept erarbeitet.

Bei Software-Projekten entspricht dieses Kapitel typischerweise der 
Analyse \& Design-Phase des \ac{rup}. Typische Ergebnisse dieser Phase sind 
Klassendiagramme etc.

\section{Verwendete Technologien}
\label{sec:verwendete_technologien}

\begin{itemize}
    \item Wahl des Backend Frameworks  und der Sprache
    \begin{itemize}
        \item Probleme mit MongoDB Client in Java/Spring
        \item Python und Flask sind lightweight
    \end{itemize}
    \item Wahl des Frontend Frameworks und der Sprache
    \begin{itemize}
        \item bestehendes Projekt mit diesem Framework
        \item Verbreitung von React
        \item Modularität dank React
        \item Warum Web?
    \end{itemize}
\end{itemize}

\section{Bestehende Visualisierungstools}
\label{sec:bestehende_visualisierungstools}

\section{Analyse der MongoDB Datenbank}
\label{sec:mongoDB_analyse}

\section{Planung des Frontends}
\label{sec:planung_frontend}

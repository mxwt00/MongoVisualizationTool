\iffalse
Auf der Basis der im vorangegangenen Kapitel erstellten Problemanalyse 
und der im Grundlagenkapitel aufgearbeiteten theoretischen Kenntnisse 
wird ein Lösungskonzept erarbeitet.

Bei Software-Projekten entspricht dieses Kapitel typischerweise der 
Analyse \& Design-Phase des \ac{rup}. Typische Ergebnisse dieser Phase sind 
Klassendiagramme etc.
\fi

% TODO wo soll diese Sektion hin?
\section{Bestehende Visualisierungstools}
\label{sec:bestehende_visualisierungstools}

Es gibt bereits MongoDB Visualisierungtools auf dem Markt, jedoch erfüllt keines davon die zuvor definierten Anforderungen zu genüge:

\begin{itemize}
    \item \textbf{MongoDB Data Explorer} ist ein Tool, welches in MongoDB Atlas integriert ist.
        MongoDB Atlas ist ein Web-Tool zur verwaltung von MongoDB Datenbanken.
        Mittels dem MongoDB Data Explorer kann man die Dokumente, Collections und Indexe einer Datenbank anschauen, sowie die Daten mit CRUD Operationen verwalten.
        Jedoch bietet der MongoDB Data Explorer keine Möglichkeiten, die Schemas der Dokumente zu analysieren.
    \item \textbf{MongoDB Compass} ist eine Desktop Anwendung zur Analyse von MongoDB Datenbanken.
        MongoDB Compass besitzt ein Schema-Visualisierungstool.
        Dieses Schema-Visualisierungstool zeigt sehr genaue Daten zu jedem Feld an, ist jedoch nicht besonders übersichtlich, wenn man das gesamte Schema der Dokumente einer Collection analysieren will.
        Ebenfalls nicht gut ersichtlich in MongoDB Compass ist die Varianz im Schema zwischen den Dokumenten in einer Collection.
    \item \textbf{MongoDB Charts, Tableau, Qlik und Looker} sind Tools, die aus MongoDB Daten Graphen generieren und dadurch die Daten in einer MongoDB visualisieren können.
        Diese Tools konzentrieren sich jedoch alle auf die Visualisierung der Daten, nicht die Analyse und Visualisierung der Schemas der Dokumente.
\end{itemize}
~\autocite{knowi:mongo_vis_tools}

\section{Verwendete Technologien}
\label{sec:verwendete_technologien}

\subsection{Backend Technologien}
\label{sec:verwendete_technologien_backend}

Da die Analyse der MongoDB Dokumente sehr rechenintensiv ist, wird die Analyse in ein Backend ausgelagert.
Um den zuvor definierten Anforderungen gerecht zu werden, ist es wichtig, ein geeignetes Backend-Framwork auszuwählen.
Das Er Modellierungstool nutzt Java Spring als Backend-Framework.
Da man zum Teil Code von dem bestehenden Backend übernehmen könnte, bietet es sich deshalb an, in diesem Projekt ebenfalls Spring zu verwenden.

Für Spring gibt es eine MongoDB Implementierung namens Spring Data MongoDB.
Diese Implementierung ist jedoch dafür ausgelegt, \ac{pojo}s auf Dokumente zu mappen.
Im MongoDB Visualisierungstool sollen hingegen MongoDB Dokumente dynamisch eingelesen und analysiert werden.
Um ~\nameref{itm:a5} zu erfüllen, ist es desweiteren nötig, beliebig viele verschiedene Datenbanken gleichzeitig zu verbinden und zu analysieren.
Die Verbindung mit MongoDB Datenbanken Spring Data MongoDB erfolgt jedoch mit festgelegten Datenbanken, welche in der application.properties Datei definiert werden.
~\autocite{spring:spring-data-mongodb}
Deshalb ist die Spring Data MongoDB Bibliothek für diese Anwendung nicht geeignet.
Neben der Spring Data MongoDB Bibliothek gibt es auch noch einen anderen MognoDB Java Client, Java Sync.
Dieser funktioniert jedoch nicht zusammen mit dem Spring Framework.
Aus diesem Grund kann Spring sowie andere Java Backend Frameworks nicht genutzt werden.

Als alternatives Backend Framework mit REST API bietet sich Flask an.
Ein großer Vorteil von Flask ist, dass Flask sehr minimal ist und nur mit dem minimum an benötigten Bibliotheken vorkonfiguriert ist.
Spring ist im Gegensatz dazu ein sehr mächtiges Framework mit vielen Features, von denen in diesem Projekt aber nur sehr wenige gebraucht werden.
Ein weiterer Vorteil von Flask sowie von Python ist die Schlankheit des Codes.
In Python lässt sich meist die gleiche Funktionalität in weniger Code schreiben als in Java.
Dazu kommt, dass in Flask sehr viel weniger Boilerplate Code benötigt wird als in Spring.
Ein minimaler Endpunkt in Flask lässt sich bereits mit 2 Zeilen Code umsetzen.
Zudem ist die dynamische Typisierung in Python beim Auswerten der MongoDB Dokumente von Vorteil, da man im Voraus nicht weiß, welche Datentypen die Werte in den Dokumenten haben, und die dynamische Typisierung deshalb das Handling dieser Werte vereinfacht.
~\autocite{khoirom2020comparative}
Jedoch hat Flask nicht nur Vorteile gegenüber Spring:
Flask ist grundsätzlich deutlich unperformanter als Spring.
Dies liegt unter anderem daran, dass Python eine interpretierte Sprache ist, und Java eine kompilierte.
~\autocite{sverker:rest_comparison}
Dies widerspricht zunächst der Anforderung ~\nameref{itm:a3}.
Die Performance-Probleme lassen sich aber durch Multiprocessing ausgleichen.
Multiprocessing bedeutet, dass bestimmte Teile der Berechnung auf mehrere Threads im Prozessor aufgeteilt werden und dadurch parallell ausgeführt werden.
Python bietet eine simpel zu implementierende Lösung für Multiprocessing an, welche man bei der Analyse der Dokumente der MongoDB Datenbanken gut einsetzen kann.
Beispielsweise kann die Analyse jeder Collection von einem extra Thread ausgeführt werden.
Dadurch lässt sich die Anforderung ~\nameref{itm:a3} mit Flask erfüllen.

In Python gibt es die Bibliothek PyMongo, welche alle der genannten Nachteile von Spring Data MongoDB ausbessert:
Mit PyMongo kann man direkt im Code beliebig viele MongoDB Datenbanken parallel dynamisch einbinden.
Mittels Objektorientierung lässt sich dadurch die parallele Analyse mehrerer Datenbanken sinnvoll umsetzen.
Zudem ist PyMongo nicht für das Mappen von Dokumenten auf Objekte gedacht.
Stattdessen kann man Dokumente als Python Dictionary auslesen.
Dies erleichtert die Analyse der Dokumente und hat darüber hinaus den Vorteil, dass man Dictionaries in Python in JSON umwandeln kann, was das Bauen der HTTP Response vereinfacht.
~\autocite{mongodb:pymongo}

\subsection{Frontend Technologien}
\label{sec:verwendete_technologien_frontend}

Web Apps haben gegenüber Desktop Apps einige Vorteile:
Web Apps müssen nicht installiert werden, sie müssen nicht für mehrere Betriebssysteme entwickelt werden und der Auslieferungs- Update- und Administrierungsprozess ist deutlich vereinfacht.
Jedoch haben Web Apps oftmals nicht die interaktionsmöglichkeiten von Desktop Apps, da sie innerhalb eines Browsers laufen.
Dies ist in dieser Anwendung jedoch kein großer Nachteil, da die Hauptaufgabe der Anwendung die Visualisierung von Daten ist, und dies nicht viele Interaktionsmöglichkeiten erfordert.
~\autocite{zepeda2007desktop}
Deshalb wird das Frontend dieser Anwendung als Webapp entwickelt.

Das ER Modellierungstool benutzt das Frontend Framework React.
Da das ER Moddelierungstool und das MongoDB Visualisierungstool Teil eines Datenbank Toolkits werden sollen, muss das MongoDB Visualisierungstool Frontend ebenfalls in React geschrieben werden, damit Anforderung ~\nameref{itm:a1} erfüllt werden kann.
Da React dank React Elements und React Components in seiner Grundstruktur  sehr modular ist, eignet sich React sehr gut, um Anforderung ~\nameref{itm:a2} zu erfüllen.
~\autocite{banks:react}
Die Tools lassen sich mittels Ordner strukturell voneinander Trennen, und trotzdem können die Tools sich Komponenten teilen und diese wiederverwenden.
Dank der Komponentenbibliothek Material UI kann man in React vordefinierte Elemente benutzen, was an vielen das Definieren von Komponenten von Hand erspart.
Dadurch spart man sich einerseits Programmieraufwand, andererseits verringert dies aber auch die Code-Komplexität und verbessert somit die Lesbarkeit des Codes.
~\autocite{mui:mui}

\section{Analyse der MongoDB Datenbank}
\label{sec:mongoDB_analyse}

\section{Planung des Frontends}
\label{sec:planung_frontend}

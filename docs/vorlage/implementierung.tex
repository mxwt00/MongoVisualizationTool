In diesem Kapitel wird die konkrete Implementierung des im Kapitel
\ref{cha:loesungskonzept} entwickelten Lösungskonzepts beschrieben.
Hierbei wird auf die konkret verwendeten Entwicklungswerkzeuge etc. 
Bezug genommen.

Bei Software-Projekten besteht dieses Kapitel typischerweise aus den 
Phasen Implementierung \& Test im \ac{rup}.

Zum Beispiel kann man hier auch ein kleines Listing einfügen.

\begin{lstlisting}[language=c,%
                   caption={Überschrift des Quelltexts}]
#include<stdio.h>

int main() {
    // Kommentar
    int answer = 20 << 1;
    answer += 2;
    printf("Hallöchen Welt!\n");
    printf("Die Antwort ist: %d\n", answer);
    return 0;
}
\end{lstlisting}

Manchmal hilft auch eine kleine Tabelle:

\begin{table}[htbp]
\centering
\begin{tabular}{|l|r|}
\hline
\textbf{Messwert a} & \textbf{Messwert b} \\ \hline
9 & 5 \\ \hline
1 & 4 \\ \hline
1 & 3 \\ \hline
\end{tabular}
\caption{Überschrift der Tabelle}
\label{tab:my-table}
\end{table}

Details siehe Tabelle~\ref{tab:my-table}.

\section{Backend}
\label{sec:backend}

\subsection{Analyse der MongoDB Datenbank}
\label{sub:mongoDB_analyse}

\subsection{REST API}
\label{sub:rest_api}

\section{Frontend}
\label{sec:frontend}

\subsection{}
\label{sub:rest_api}


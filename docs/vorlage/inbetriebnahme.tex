\iffalse
Aufgabe des Kapitels Inbetriebnahme ist es, die Überführung der in 
Kapitel \ref{cha:implementierung} entwickelte Lösung in das betriebliche 
Umfeld aufzuzeigen. Dabei wird beispielsweise die Inbetriebnahme eines 
Programms beschrieben oder die Integration eines erstellten 
Programmodules dargestellt.

Bei der Software-Erstellung entspricht dieses Kapitel der 
Auslieferungsphase (Deployment) im \ac{rup}.
\fi

Das MongoDB Visualisation Tool wird auf einem Linux Server mit Debian 11 gehostet.
Das Flask Backend läuft in einem Docker Container.
Der Vorteil von Docker ist, dass der Applikation eine eigene Umgebung bereitgestellt wird, die unabhängig vom Host-System ist.
Um das Backend als Dockercontainer auszuliefern, wurde das vorkonfigurierte Docker Image uwsgi-nginx-flask von tiangolo genutzt.
In diesem Docker Image ist uWSGI und Nginx vorinstalliert, was das ausliefern von Flask Applikationen erleichtert.
~\autocite{tiangolo:uwsgi-nginx-flask}

Das Backend für das ER Modelling Tool ist als Heroku-App deployed.
Heroku ist ein cloudbasierter Serviceanbieter, welcher ein kostenloses Modell für das Hosting von Software und DNS anbietet.

Das Frontend wurde mit NPM gebaut und auf dem Server bereitgestellt.
Mittels NGINX wurde das Webinterface daraufhin gehostet. 
Das Interface ist nun unter der Adresse \url{http://kodewisch.com:12030} erreichbar.

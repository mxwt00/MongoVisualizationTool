\iffalse
Aufgabe des Kapitels Inbetriebnahme ist es, die Überführung der in 
Kapitel \ref{cha:implementierung} entwickelte Lösung in das betriebliche 
Umfeld aufzuzeigen. Dabei wird beispielsweise die Inbetriebnahme eines 
Programms beschrieben oder die Integration eines erstellten 
Programmodules dargestellt.

Bei der Software-Erstellung entspricht dieses Kapitel der 
Auslieferungsphase (Deployment) im \ac{rup}.
\fi

Das MongoDB Visualisation Tool wird auf einem Linux Server mit Debian 11 gehostet.
Das Flask Backend läuft in einem Docker Container.
Der Vorteil von Docker ist, dass der Applikation eine eigene Umgebung bereitgestellt wird, die unabhängig vom Host-System ist.
Um das Backend als Dockercontainer auszuliefern, wurde das vorkonfigurierte Docker Image uwsgi-nginx-flask von tiangolo genutzt.
In diesem Docker Image sind uWSGI und Nginx vorinstalliert, was das Ausliefern von Flask Applikationen erleichtert.
~\autocite{tiangolo:uwsgi-nginx-flask}

Das Backend für das ER Modelling Tool ist als Heroku-App deployed.
Heroku ist ein cloudbasierter Serviceanbieter, welcher ein kostenloses Modell für das Hosting von Software und DNS anbietet.
~\autocite{heroku:heroku}

Das Frontend wurde mittels NPM gebaut und auf dem Server bereitgestellt.
Mittels NGINX wurde das Webinterface daraufhin gehostet. 
Das Interface ist nun unter der Adresse \url{https://db-toolkit.kodewisch.com} erreichbar.

\section{Buildprozess des Backends}
\label{sec:build_backend}

Um das Backend starten zu können, muss Python 3.10 oder neuer installiert sein.

Wenn dies der Fall ist, können mit dem Befehl \lstinline{pip -r requirements.txt} im Verzeichnis mongo\_vis\_backend alle benötigten Pakete installiert werden.
Um Konflikte mit anderen Python Installationen zu vermeiden, führt man diesen Befehl am besten in einem Virtual Environment aus.
Anschließend kann das Backend mit dem Befehl \lstinline{python3 -m flask run} ausgeführt werden.

Eine Anleitung für das Deployment des Backends als Docker Container kann unter \url{https://hub.docker.com/r/tiangolo/uwsgi-nginx-flask/} gefunden werden.

\section{Buildprozess des Frontends}
\label{sec:build_frontend}

Node und NPM müssen installiert sein, damit das Frontend gebaut werden kann.
Über \url{https://nodejs.org/en/download/} können Node und NPM installiert werden.

Anschließend müssen im Verzeichnis frontend folgende Befehle ausgeführt werden, um das Frontend zu starten:

\begin{lstlisting}[language=bash, caption={Frontend build commands},label={lst:fe_build_commands}]
$ npm install
$ npm start
\end{lstlisting}    

Um das Frontend auszuliefern, kann das Frontend mit dem Befehl \lstinline{npm run build} gebaut werden.
Die gebauten Dateien befinden sich dann im Ordner build.

Vorausgesetzt wird ein Debian- oder Ubuntubasierter Computer oder Server, auf dem Nginx mit SSL-Zertifikat konfiguriert ist.
Um die Dateien auf solch einem Server auszuliefern, müssen die Dateien in den Ordner /var/www/db-toolkit/html verschoben werden.
Der Ordnername db-toolkit kann auch durch einen beliebigen anderen Namen ersetzt werden.
Daraufhin muss folgende Nginx Konfiguration im Ordner /etx/nginx/conf.d mit dem namen db-toolkit.conf angelegt werden:

\begin{lstlisting}[language=bash, caption={db-toolkit.conf},label={lst:db-toolkit.conf}]
server { 
    listen 80; 
    listen [::]:80; 
    server_name db-toolkit.kodewisch.com; 
    return 301 https://$host$request_uri; 
} 

server { 
    listen 443 ssl; 
    server_name db-toolkit.kodewisch.com;

    ssl_certificate /etc/letsencrypt/live/kodewisch.com/fullchain.pem; 
    ssl_certificate_key /etc/letsencrypt/live/kodewisch.com/privkey.pem; 
    include /etc/letsencrypt/options-ssl-nginx.conf; 
    ssl_dhparam /etc/letsencrypt/ssl-dhparams.pem; 

    root /var/www/db-toolkit/html;
    index index.html index.htm;

    location / { 
        try_files $uri $uri/ =404; 
    } 
}
\end{lstlisting}    

Diese Nginx Konfiguration zeigt die unter root und index spezifizierte Resource für alle Anfragen an, die auf die Subdomain db-toolkit.kodewisch.com eingehen.
Die obere Server-Konfiguration nimmt die HTTP Anfragen auf Port 80 entgegen und leitet diese auf HTTPS um.
Die untere Server-Konfiguration nimmt die HTTPS Anfragen auf Port 443 entgegen, spezifiziert das SSL-Zertifikat und zeigt die Resource an.
Die ssl\_certificate Pfade müssen durch die entsprechenden Pfade auf dem Server ersetzt werden, ebenso wie die Domain.
